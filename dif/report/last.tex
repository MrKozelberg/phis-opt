% document type
\documentclass[12pt]{article}

% packages
\usepackage[total={170mm,230mm}]{geometry}
\usepackage[utf8]{inputenc}
\usepackage[T1]{fontenc}
\usepackage[russian]{babel}
\usepackage{graphicx}
\usepackage{amssymb}
\usepackage{amsfonts}
\usepackage{amsmath}
\usepackage{amsthm}
\usepackage{physics}
\usepackage{nicefrac}
\usepackage{cancel}
\usepackage{hyperref}
\usepackage{cmap}
\usepackage{textcomp}

\title{Дифракционная решетка}
\author{Алексей Чубаров \and Козлов Александр}

\begin{document}
\maketitle

Измерили координаты максимумов дифракционной картины для длины волны $\lambda = 450\ \text{нм}$. Результаты отобразили в таблице \ref{tab:1}.
\begin{table}[htbp]
\centering
\caption{Результаты измерений координат максимумов дифракционной картины для длины волны $\lambda = 450\ \text{нм}$.}
\begin{tabular}{|r|r|r|r|r|} 
\hline
\multicolumn{1}{|l|}{$\# \text{max}$} & \multicolumn{1}{l|}{$\theta^\circ$} & \multicolumn{1}{l|}{d, нм} & \multicolumn{1}{l|}{$\delta d$} & \multicolumn{1}{l|}{$\Delta d,\ \text{нм}$}  \\ 
\hline
3                                              & 56                                  & 1,64E+03                   & 0,04                            & 7E+01                                        \\ 
\hline
2                                              & 34                                  & 1,6E+03                    & 0,06                            & 1E+02                                        \\ 
\hline
1                                              & 16                                  & 1,6E+03                    & 0,11                            & 2E+02                                        \\ 
\hline
-1                                             & -16                                 & 1,7E+03                    & 0,12                            & 2E+02                                        \\ 
\hline
-2                                             & -33                                 & 1,7E+03                    & 0,06                            & 1E+02                                        \\ 
\hline
-3                                             & -55                                 & 1,66E+03                   & 0,04                            & 7E+01                                        \\
\hline
\end{tabular}
\label{tab:1}
\end{table}
\par Провели аналогичное измерение для красного цвета (использовался гелий-неоновый лазер с $\lambda = 710\ \text{нм}$), их результаты выведены на таблицу 
\begin{table}[htbp]
\centering
\caption{Результаты измерений координат максимумов дифракционной картины для длины волны $\lambda = 710\ \text{нм}$.}
\begin{tabular}{|r|r|r|r|r|} 
\hline
\multicolumn{1}{|l|}{$\# \text{max}$} & \multicolumn{1}{l|}{$\theta^\circ$} & \multicolumn{1}{l|}{d, нм} & \multicolumn{1}{l|}{$\delta d$} & \multicolumn{1}{l|}{$\Delta d,\ \text{нм}$}  \\ 
\hline
2                           & 59                      & 7,1E+02                    & 0,09                    & 6E+01                        \\ 
\hline
1                           & 25                      & 7E+02                      & 0,17                    & 1E+02                        \\ 
\hline
-1                          & -27                     & 7E+02                      & 0,17                    & 1E+02                        \\ 
\hline
-2                          & -60                     & 7,2E+02                    & 0,09                    & 6E+01                        \\
\hline
\end{tabular}
\end{table}
\par Из данных таблиц видно, что по углу третий максимум для $\lambda = 450\ \text{нм}$ совпадает со вторым максимумом для $\lambda = 710\ \text{нм}$. Таким образом дисперсионная область будет
\begin{equation}
	\Delta \lambda = 710 - 450 = 260\ \text{нм}.
\end{equation}
Абсолютная погрешность будет
\begin{equation}
	\Delta\qty(\Delta \lambda) = 70\ \text{нм}.
\end{equation}
Относительная погрешность будет
\begin{equation}
	\delta \qty(\Delta \lambda) = 27\%.
\end{equation}
Исходя из теории дисперсионная область есть
\begin{equation}
	\Delta \lambda = \dfrac{\lambda}{m} = \dfrac{450}{2} = 225\ \text{нм}.
\end{equation}
Что с учётом погрешностей совпадает с вычисленным ранее.
\end{document}